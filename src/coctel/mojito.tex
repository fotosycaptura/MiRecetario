% Mojito
\newpage
%\thispagestyle{empty}
\begin{recipe}[source={Internet},
	portion={1 porción},
	preparationtime={\unit[1]{Minuto}}
	]{Mojito}
	%	\includegraphics[width=0.25\textwidth]{Mojito}
	\introduction{
		Esta receta la vi por internet...
	}
	\ingredients{
		2 & Cucharaditas de azúcar \\
		8 & hojas de menta \\
		30 & \unit{ml} de limón \\
		60 & \unit{ml} de ron \\
		3 o 4 & rodajas de limón \\
		120 & \unit{ml} de soda \\
		& Hielo picado \\		
	}
	\preparation{
		\begin{enumerate}
			\item Gran parte del secreto de un mojito es la menta.
			\item Se agrega el azúcar al fondo del vaso. Los mojitos se elaboran directamente sobre el vaso, sin necesidad de coctelera. Se vierte el jugo del limón y con la ayuda de un mortero o con una cuchara de bar, se diluye el limón con el azúcar.
			\item Se machaca ligeramente las hojas de menta para que liberen su aroma apretándolas contra el azúcar en el fondo. No deben quedar totalmente rotas y machacadas, porque entonces el mojito resulta desagradable de beber. Por eso \textit{ligeramente}.
			\item Se agregan las rodajas de limón en el fondo del mojito y se le da unos toques de mortero para que libere un poco de su jugo.
			\item Verter el ron y llenar el vaso con hielo picado. Utilizar abundante hielo  lo hace más fresco. Rellenamos el cóctel con soda hasta completar, unas gotas de angostura (opcionales) y revolver con suavidad. 
			\item Servir y ¡a disfrutar!
		\end{enumerate}
	}
	%	\hint{
		%		
		%	}
\end{recipe}