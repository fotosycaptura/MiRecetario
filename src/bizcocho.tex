% Bizcocho
\newpage
\thispagestyle{empty}
\begin{recipe}[source={Esbieta},
preparationtime={\unit[1]{Horas}\unit[25]{Minutos}}
]{Bizcocho Para Torta}
\introduction{
Siempre pensé que el bizcocho para torta no debería de separarse la clara de las yemas para ir preparando, y encontré esta receta que es más antigua y funciona muy bien
}
\ingredients{
	7 & huevos \\
    \unit[210]{gr} & de harina \\
    \unit[210]{gr} & de azúcar \\
    1 & cucharadita de sal \\
    1 & ralladura de limón
}
\preparation{
    \begin{enumerate}
        \item Cascar los huevos en un bol, agregar el azúcar y proceder a mojar el azúcar con los huevos para que no salga volando al batir.
        \item En otro bol, tamizar la harina.
        \item Batir fuertemente por aproximadamente entre 12 a 15 minutos. Sabrá cuando está montada cuando al trazar una línea, ésta no desaparece después de 8 conteos. Desde este punto, se puede mezclar cuidadosamente con la harina.
        \item Con movimientos envolventes se debe de mezclar la harina con la mezcla cuidadosamente.
        \item Precalentar el horno por 10 minutos a 170 grados con calor arriba y abajo.
        \item Verter la mezcla en un molde previamente enmantequillado o forrado con papel mantequilla.
        \item Llevar al horno y dejar cocinar por aproximadamente 30-40 minutos. Es muy importante no abrir el horno durante al menos 20 minutos. Transcurrido los 30-35 minutos se puede verificar con un palillo que el bizcocho esté completamente cocido.
        \item Si el bizcocho está bien, se puede retirar con seguridad del horno para dejar enfriar.
    \end{enumerate}
}
\hint{
	Al momento de rellenar con manjar cada lonja, se puede remojar con una solución de 2 cucharadas de vino dulce blanco en 1 taza con agua tibia
}
\end{recipe}